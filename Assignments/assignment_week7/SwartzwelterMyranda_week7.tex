% Options for packages loaded elsewhere
\PassOptionsToPackage{unicode}{hyperref}
\PassOptionsToPackage{hyphens}{url}
%
\documentclass[
]{article}
\title{Student Survey}
\author{Myranda Swartzwelter}
\date{1/25/2022}

\usepackage{amsmath,amssymb}
\usepackage{lmodern}
\usepackage{iftex}
\ifPDFTeX
  \usepackage[T1]{fontenc}
  \usepackage[utf8]{inputenc}
  \usepackage{textcomp} % provide euro and other symbols
\else % if luatex or xetex
  \usepackage{unicode-math}
  \defaultfontfeatures{Scale=MatchLowercase}
  \defaultfontfeatures[\rmfamily]{Ligatures=TeX,Scale=1}
\fi
% Use upquote if available, for straight quotes in verbatim environments
\IfFileExists{upquote.sty}{\usepackage{upquote}}{}
\IfFileExists{microtype.sty}{% use microtype if available
  \usepackage[]{microtype}
  \UseMicrotypeSet[protrusion]{basicmath} % disable protrusion for tt fonts
}{}
\makeatletter
\@ifundefined{KOMAClassName}{% if non-KOMA class
  \IfFileExists{parskip.sty}{%
    \usepackage{parskip}
  }{% else
    \setlength{\parindent}{0pt}
    \setlength{\parskip}{6pt plus 2pt minus 1pt}}
}{% if KOMA class
  \KOMAoptions{parskip=half}}
\makeatother
\usepackage{xcolor}
\IfFileExists{xurl.sty}{\usepackage{xurl}}{} % add URL line breaks if available
\IfFileExists{bookmark.sty}{\usepackage{bookmark}}{\usepackage{hyperref}}
\hypersetup{
  pdftitle={Student Survey},
  pdfauthor={Myranda Swartzwelter},
  hidelinks,
  pdfcreator={LaTeX via pandoc}}
\urlstyle{same} % disable monospaced font for URLs
\usepackage[margin=1in]{geometry}
\usepackage{longtable,booktabs,array}
\usepackage{calc} % for calculating minipage widths
% Correct order of tables after \paragraph or \subparagraph
\usepackage{etoolbox}
\makeatletter
\patchcmd\longtable{\par}{\if@noskipsec\mbox{}\fi\par}{}{}
\makeatother
% Allow footnotes in longtable head/foot
\IfFileExists{footnotehyper.sty}{\usepackage{footnotehyper}}{\usepackage{footnote}}
\makesavenoteenv{longtable}
\usepackage{graphicx}
\makeatletter
\def\maxwidth{\ifdim\Gin@nat@width>\linewidth\linewidth\else\Gin@nat@width\fi}
\def\maxheight{\ifdim\Gin@nat@height>\textheight\textheight\else\Gin@nat@height\fi}
\makeatother
% Scale images if necessary, so that they will not overflow the page
% margins by default, and it is still possible to overwrite the defaults
% using explicit options in \includegraphics[width, height, ...]{}
\setkeys{Gin}{width=\maxwidth,height=\maxheight,keepaspectratio}
% Set default figure placement to htbp
\makeatletter
\def\fps@figure{htbp}
\makeatother
\setlength{\emergencystretch}{3em} % prevent overfull lines
\providecommand{\tightlist}{%
  \setlength{\itemsep}{0pt}\setlength{\parskip}{0pt}}
\setcounter{secnumdepth}{-\maxdimen} % remove section numbering
\ifLuaTeX
  \usepackage{selnolig}  % disable illegal ligatures
\fi

\begin{document}
\maketitle

\hypertarget{student-survey-covariance}{%
\subsection{Student Survey Covariance}\label{student-survey-covariance}}

Covariance is the averaged sum of combined deviations. A positive
covariance indicates that as one variable deviates from the mean, the
other variable also deviates from the mean in the same direction. A
negative covariance indicates that as one variable deviates from the
mean, the other variable deviates from the mean in the other
direction.The magnitude of difference is dependent on the scale of the
data being measured.

\begin{longtable}[]{@{}lrrrr@{}}
\caption{Covariance of Student Survey variables}\tabularnewline
\toprule
& TimeReading & TimeTV & Happiness & Gender \\
\midrule
\endfirsthead
\toprule
& TimeReading & TimeTV & Happiness & Gender \\
\midrule
\endhead
TimeReading & 3.0545455 & -20.3636364 & -10.350091 & -0.0818182 \\
TimeTV & -20.3636364 & 174.0909091 & 114.377273 & 0.0454545 \\
Happiness & -10.3500909 & 114.3772727 & 185.451422 & 1.1166364 \\
Gender & -0.0818182 & 0.0454545 & 1.116636 & 0.2727273 \\
\bottomrule
\end{longtable}

These results indicate that:

\begin{itemize}
\tightlist
\item
  TimeReading is inversely related to TimeTV, Happiness and Gender and
  vice versa.
\item
  Time TV is positively related to Happiness and Gender and vice versa.
\item
  Happiness is positively related to Gender and vice versa.
\end{itemize}

\hypertarget{examining-variables}{%
\subsection{Examining Variables}\label{examining-variables}}

\begin{longtable}[]{@{}rrrr@{}}
\caption{Example of Student Survey Variables}\tabularnewline
\toprule
TimeReading & TimeTV & Happiness & Gender \\
\midrule
\endfirsthead
\toprule
TimeReading & TimeTV & Happiness & Gender \\
\midrule
\endhead
1 & 90 & 86.20 & 1 \\
2 & 95 & 88.70 & 0 \\
2 & 85 & 70.17 & 0 \\
2 & 80 & 61.31 & 1 \\
3 & 75 & 89.52 & 1 \\
4 & 70 & 60.50 & 1 \\
\bottomrule
\end{longtable}

Looking at the variables in the student survey table, we find that all
of the variables are using different scales of measurement. For example,
it appears that TimeReading is likely measured in hours whereas TimeTV
is likely measured in minutes. Although if corrected for this by
converting TimeReading to minutes or TimeTV to hours, it would not
change the sign of the covariance (so the relationship between variables
remains positive or inverse), it would change the magnitude of the
covariance between variables

\begin{longtable}[]{@{}lrrrrr@{}}
\caption{Covariance of Student Survey variables}\tabularnewline
\toprule
& TimeReading & TimeReading\_min & TimeTV & Happiness & Gender \\
\midrule
\endfirsthead
\toprule
& TimeReading & TimeReading\_min & TimeTV & Happiness & Gender \\
\midrule
\endhead
TimeReading & 3.0545455 & 183.272727 & -20.3636364 & -10.350091 &
-0.0818182 \\
TimeReading\_min & 183.2727273 & 10996.363636 & -1221.8181818 &
-621.005455 & -4.9090909 \\
TimeTV & -20.3636364 & -1221.818182 & 174.0909091 & 114.377273 &
0.0454545 \\
Happiness & -10.3500909 & -621.005455 & 114.3772727 & 185.451422 &
1.1166364 \\
Gender & -0.0818182 & -4.909091 & 0.0454545 & 1.116636 & 0.2727273 \\
\bottomrule
\end{longtable}

Happiness is likely on a scale from 1-100 but we have no concept of a
scale for this metric. It is unable to be changed into a minute or hour
measure like the previous variables were.

Gender is an indicator function, indicating a student is on gender if
they are labeled as `1' and the other gender if labeled as `0'.

If you swap the gender indicators (i.e.~the 1's become 0's and the 0's
become 1's), you also change the direction of the relationship between
Gender and the other variables, but maintain the magnitude. See table
for examples.

\begin{longtable}[]{@{}lrrrrr@{}}
\caption{Covariance of Student Survey variables}\tabularnewline
\toprule
& TimeReading & TimeTV & Happiness & Gender & Gender\_Swap \\
\midrule
\endfirsthead
\toprule
& TimeReading & TimeTV & Happiness & Gender & Gender\_Swap \\
\midrule
\endhead
TimeReading & 3.0545455 & -20.3636364 & -10.350091 & -0.0818182 &
0.0818182 \\
TimeTV & -20.3636364 & 174.0909091 & 114.377273 & 0.0454545 &
-0.0454545 \\
Happiness & -10.3500909 & 114.3772727 & 185.451422 & 1.1166364 &
-1.1166364 \\
Gender & -0.0818182 & 0.0454545 & 1.116636 & 0.2727273 & -0.2727273 \\
Gender\_Swap & 0.0818182 & -0.0454545 & -1.116636 & -0.2727273 &
0.2727273 \\
\bottomrule
\end{longtable}

TimeTV, TimeReading and Happiness are ratio variables, where as gender
is binary.

\hypertarget{type-of-correlation-analysis}{%
\subsection{Type of Correlation
Analysis}\label{type-of-correlation-analysis}}

First we want to look at the type of variables. For TimeTV, TimeReading
and Happiness, the variables are ration and linear, so we can do a
Pearson's R test. However for Gender, we must do a biserial correlation.
This test is the same in R as the Pearson's test so we'll use Pearson's
for all variables. Because of the values we saw in the covariances, I
believe:

\begin{itemize}
\tightlist
\item
  TimeReading will be negatively correlated with TimeTV, Happiness and
  Gender.
\item
  Time TV will be positively correlated to Happiness and Gender.
\item
  Happiness will be positively correlated to Gender.
\end{itemize}

\hypertarget{correlation-analysis}{%
\subsection{Correlation Analysis}\label{correlation-analysis}}

Correlations of all variables. For this analysis we'll change the
variables of time reading to be in minutes to match tv time.

\begin{longtable}[]{@{}lrrrr@{}}
\caption{Pearson R correlation of Student Survey
variables}\tabularnewline
\toprule
& TimeReading & TimeTV & Happiness & Gender \\
\midrule
\endfirsthead
\toprule
& TimeReading & TimeTV & Happiness & Gender \\
\midrule
\endhead
TimeReading & 1.0000000 & -0.8830677 & -0.4348663 & -0.0896421 \\
TimeTV & -0.8830677 & 1.0000000 & 0.6365560 & 0.0065967 \\
Happiness & -0.4348663 & 0.6365560 & 1.0000000 & 0.1570118 \\
Gender & -0.0896421 & 0.0065967 & 0.1570118 & 1.0000000 \\
\bottomrule
\end{longtable}

Correlation of two variables TimeTV and TimeReading:

\begin{verbatim}
## 
##  Pearson's product-moment correlation
## 
## data:  student_survey_df$TimeReading and student_survey_df$TimeTV
## t = -5.6457, df = 9, p-value = 0.0001577
## alternative hypothesis: true correlation is less than 0
## 95 percent confidence interval:
##  -1.0000000 -0.6684786
## sample estimates:
##        cor 
## -0.8830677
\end{verbatim}

Correlation of two variables TimeTV and TimeReading with 99\% Confidence
Interval:

\begin{verbatim}
## 
##  Pearson's product-moment correlation
## 
## data:  student_survey_df$TimeReading and student_survey_df$TimeTV
## t = -5.6457, df = 9, p-value = 0.0001577
## alternative hypothesis: true correlation is less than 0
## 99 percent confidence interval:
##  -1.0000000 -0.5131843
## sample estimates:
##        cor 
## -0.8830677
\end{verbatim}

Looking at the correlation matrix we can say:

\begin{itemize}
\tightlist
\item
  TimeTV and TimeReading are strongly negatively correlated (p value
  calculated \textless{} 0.05)
\item
  TimeTV and Happiness are potentially positively correlated
\item
  TimeTV and Gender are not correlated
\item
  TimeReading and Happiness are potentially negatively correlated
\item
  TimeReading and Gender are not correlated
\item
  Happiness and Gender may be slightly positively correlated
\end{itemize}

Based on your analysis can you say that watching more TV caused students
to read less? Explain.

Yes, when doing a correlation test, we see that the null hypothesis is
rejected with a p value of 0.000157 so TimeReading and TimeTV are
negatively correlated, or the more someone watches TV, the less time
they spend reading.

Pick three variables and perform a partial correlation, documenting
which variable you are ``controlling''. Explain how this changes your
interpretation and explanation of the results.

I am choosing to perform a partial correlation for TimeTV and
TimeReading while controlling for Happiness.

\begin{verbatim}
## [1] -0.872945
\end{verbatim}

\begin{verbatim}
## $tval
## [1] -5.061434
## 
## $df
## [1] 8
## 
## $pvalue
## [1] 0.0009753126
\end{verbatim}

When controlling for happiness, the partial correlation for TVTime and
TVReading goes from -0.88 to -0.87. This is still a strong correlation.
The p value for the partial correlation is 0.000975 which is still
significantly below 0.05. This means that even when controlling for
happiness, there is a strong negative correlation between TimeTV and
TimeReading.

\end{document}
